%%%%%%%%%%%%%%%%%%%%%%%%%%%%%%%%%%%%%%%%%
% Thin Sectioned Essay
% LaTeX Template
% Version 1.0 (3/8/13)
%
% This template has been downloaded from:
% http://www.LaTeXTemplates.com
%
% Original Author:
% Nicolas Diaz (nsdiaz@uc.cl) with extensive modifications by:
% Vel (vel@latextemplates.com)
%
% License:
% CC BY-NC-SA 3.0 (http://creativecommons.org/licenses/by-nc-sa/3.0/)
%
%%%%%%%%%%%%%%%%%%%%%%%%%%%%%%%%%%%%%%%%%

%----------------------------------------------------------------------------------------
%   PACKAGES AND OTHER DOCUMENT CONFIGURATIONS
%----------------------------------------------------------------------------------------

\documentclass[11pt]{article} % Font size (can be 10pt, 11pt or 12pt) and paper size (remove a4paper for US letter paper)

\usepackage[utf8]{inputenc} % Set utf8 code
\usepackage[protrusion=true,expansion=true]{microtype} % Better typography
\usepackage[portuguese]{babel}
\usepackage{graphicx} % Required for including pictures
\usepackage{wrapfig} % Allows in-line images
\usepackage[pagebackref]{hyperref}

\usepackage{mathpazo} % Use the Palatino font
\usepackage[T1]{fontenc} % Required for accented characters

\usepackage{wallpaper}
\usepackage[font={color=white,bf},figurename=Fig.,labelfont={it}]{caption}
\usepackage{lipsum, xcolor, etoolbox, footmisc, bigfoot}

\usepackage{tabu}
\hypersetup{
    colorlinks=false,
    pdfborder={0 0 0},
}

\setcounter{secnumdepth}{5}
\setcounter{tocdepth}{5}

\linespread{1.05} % Change line spacing here, Palatino benefits from a slight increase by default

\makeatletter
\renewcommand\@biblabel[1]{\textbf{#1.}} % Change the square brackets for each bibliography item from '[1]' to '1.'
\renewcommand{\@listI}{\itemsep=0pt} % Reduce the space between items in the itemize and enumerate environments and the bibliography

\renewcommand{\maketitle}{ % Customize the title - do not edit title and author name here, see the TITLE block below


\begin{center} % Right align
{\LARGE\@title} % Increase the font size of the title

\vspace{20pt} % Some vertical space between the title and author name

\end{center}
}

\patchcmd{\ps@plain}{\thepage}{\textcolor{white}{\thepage}}{}{}
\makeatother

\begin{document}

\ThisTileWallPaper{\paperwidth}{\paperheight}{res/wallpaper_header.jpg}
\color{white}
\pagestyle{plain}
\def\footnotelayout{\color{white}}
\renewcommand\thefootnote{\textcolor{white}{\arabic{footnote}}}
\begin{titlepage}
 \vfill
  \begin{center}
   {\textbf{{{\Huge  Strife Of Mythology Tower Defense}}}}\\[6cm]


   {{\huge Game Design Document}}\\[6cm]

   \hspace{.45\textwidth} %posiciona a minipage
  \vfill

\vspace{2cm}

\large \textbf{Brasília}

\large \textbf{Abril de 2016}
\end{center}
\end{titlepage}
\newpage
\color{black}
\tableofcontents

\newpage

%----------------------------------------------------------------------------------------
%   DOC BODY
%----------------------------------------------------------------------------------------

\TileWallPaper{\paperwidth}{\paperheight}{res/wallpaper_body.jpg}
\color{white}

\section*{Tabela de Revisão}


\begin{table}[h]

  \taburulecolor{white}
  \color{white}
\begin{tabu}{|l|l|p{60mm}|l|}

\hline 
\textbf{Versão}     & \textbf{Data}     & \textbf{Descrição}                              			& \textbf{Autor}    \\ \hline
0.1                 & 05/03/16        & Objetivo / História / Controles                    			& Marcelo Martins     \\ \hline
0.2                 & 06/03/16        & Requisitos Tecnológicos                            			& Marcelo Martins     \\ \hline
0.3                 & 13/03/16        & Aplicando correções propostas                      			& Marcelo Martins     \\ \hline
0.4                 & 12/04/16      & Adição dos Personagens / Refinamento dos itens anteriores   	& Marcelo Martins     \\ \hline
0.5                 & 29/04/2015        & Rascunho das telas				                   			& Jônntas Lennon     \\ \hline
0.6                 & 10/05/2015        & telas / Personagens				                   			& Jônntas Lennon     \\ \hline
0.7                 & 18/05/2015        & Economia				                   			& Jônntas e Marcelo     \\ \hline
0.8                 & 18/05/2015        & Personagens				                   			& Jônntas e Marcelo     \\ \hline


\end{tabu}
\end{table}

\newpage

\section{Objetivo do jogo}

O \textit{\textbf{SoMTD}} é um jogo de estilo\textit{ Tower Defense}, onde o objetivo é impedir que as \textit{waves} de monstros avancem pela caminho utilizando torres que atacam estas \textit{waves}, estas torres são pertencentes a três Deuses.
 A medida em que as \textit{waves} são vencidas, elas ficam mais fortes e difíceis de derrotar. Nisto haverá um recurso que servirá para evoluir e construir novas torres, o qual será obtido conforme os monstros forem derrotados.

\section{Glossário}
\begin{itemize}
\item \textbf{Wave}: Inimigos que nascem em conjunto em momentos do jogo.
\item \textbf{Caminho}: Caminho pré-determinado que será seguido pelas \textit{waves}. Caso um inimigo chegue ao fim da trilha, o jogador sofre uma punição.
\item \textbf{Tower Defense}: Um estilo de jogo onde um jogador deve impedir unidades inimigas de completarem seu objetivo utilizando torres.
\item \textbf{Health Point(HP)}: Atributo referente a vida \textit{player}.
\item \textbf{Def} - Defense : refere-se neste contexto, aos pontos de ataques ao monstro, a defesa aqui é o nível de resistência a ataques comuns.
\item \textbf{Sp} - Speed : refere-se a velocidade com que as waves se movem.
\item \textbf{gold}: única remuneração do jogo.
\end{itemize} 
 
\section{Conceito do jogo}

As torres pertencem  três Deuses sendo eles:
\begin{itemize}
\item Hades.
\item Zeus.
\item Poseidom.
\end{itemize}

Cada Deus terá um conjunto de Torres a seu dispor, divididas em 2 categorias sendo que cada categoria possuirá uma singularidade em relação as demais, possuindo assim vantagens em relação a alguns tipos de monstros e desvantagens em relação a outros tipos de monstros.

Ao iniciar o jogo, o jogador terá uma quantidade de ouro suficiente para adquirir e colocar torres de forma estratégica, tentando evitar a passagem dos inimigos pela trilha, sendo que no início do jogo apenas um tipo de torre por Deus estará disponível, tais torres podem ser evoluídas ao passo que a torre selecionada consiga destruir os monstros. 

A mudança de cenário ocorrerá ao decorrer do jogo, a medida em que as \textit{waves} forem derrotadas o jogador alcançará pontos de performance, que ao completar a barra referente a estes pontos, ocorrerá uma troca de era, havendo assim a mudança de fase e consequentemente de cenário. 

O final do jogo pode-se dar de duas formas:

\begin{enumerate}
\item O jogador posicionou as torres da forma que ele acreditava ser a forma mais estratégica, contudo não era a melhor forma, e ele não conseguiu suportar a \textit{wave},fazendo com que os inimigos cheguem ao final da trilha, com isso ele irá acumular penalidades, atingindo o máximo de penalidades aparecera a mensagem de \textit{Game Over};
\item O jogador consegue posicionar as torres da forma mais estratégica, conseguindo assim, suportar todas as \textit{waves} impostas, caso isso ocorra, ele irá receber uma mensagem de \textit{You Win}.

\end{enumerate}

\section{Gameplay}

\subsection{Progressão do Jogo}

\subsection{Timer}

\newpage

\section{Mecânicas do Jogo}
\subsubsection{Movimentação}
A movimentação das \textit{waves} será em um caminho pré-determinado, cabendo ao jogador apenas posicionar as torres com o mouse próxima a esse caminho, não havendo assim movimentação por parte do jogador principal, caso não seja possível a inserção da torre no local desejado, como a trilha das \textit{waves}, ou um \textit{tiler} já ocupado por outra torre, tem-se um indicativo da impossibilidade de inserção da torre no local.

\subsubsection{Resolução}
A resolução do jogo será de 1124x700 \textit{pixels}, esta resolução deve dar um tamanho confortável para o jogador que terá a disposição um mapa relativamente grande.

Os monstros terão uma resolução que pode variar bastante dependo do artista que desenvolver o protótipo para tal monstro, sendo aconselhável que tenha aproximadamente 30 \textit{pixels} de largura e 50 \textit{pixels} de largura, porém este valor é relativo uma vez que os monstros possuem características físicas diferentes.

Para as torres definimos algo entre 80 a 100 \textit{pixels} de altura e 50 a 70 \textit{pixels} de largura, porém se o artista achar necessário fugir um pouco deste \textit{range}, para melhorar a caracterização do personagem e permanecendo dentro do \textit{Tile}, poderá haver esta mudança.

\subsection{Checkpoints}
O jogo será direto, não havendo \textit{checkpoints} ou seja uma vez morto o jogador terá que reiniciar no inicio da fase, e ao sair do jogo perde-se todo o progresso salvo, porém caso o mesmo avance de fase, os níveis abertos estarão disponíveis para o jogo.

\subsection{Mecânica das Torres}

As torres terão \textit{range} que é um campo de ataque, no qual somente será possível causar dano no monstro caso este esteja no campo de ação da torre.

Como dito na secção Inimigos, quando um inimigo entrar no \textit{range} das torres, as mesmas começarão a atacar os monstros obedecendo a ordem de entradas destes monstros, ou seja a torre continuara atacando o monstro até que este saia do \textit{range} de ataque ou até que o monstro esteja morto, assim atacando um novo monstro somente após estas duas condições.

\begin{figure}[!htp]
\centering
\includegraphics[scale=0.35]{res/atac.png} \quad
\includegraphics[scale=0.3]{res/dont.png}
\caption{Range de Ataque das torres, na primeira a torre ataca somente o primeiro a entrar no range, na segunda imagem a torre não ataca já que o inimigo encontra-se fora do range}
\label{Tela Barracks}
\end{figure}

\newpage

\section{Câmera e HUD}

\subsection{Câmera}

A câmera do jogo será 2D estática em um mapa isométrico, como exemplificado pela imagem abaixo.
\begin{figure}[!htp]
\centering
\includegraphics[scale=0.25]{res/map_example.png}
\caption{Tela Câmera}
\label{Tela Barracks}
\end{figure}

\subsection{HUD}

O HUD (Heads-Up Display) será composto de um menu inferior o qual será dividido em três secções, sendo elas:
\begin{itemize}
\item Um box contendo os três Deuses, ao clicar no Deus, atualizara o menu central que contem as três torres, duas de \textit{Tiler 1} e uma de \textit{Tiler 2}, exibindo os valores referentes as torres deste deus.
\item Continuando no box inferior esquerdo, tem-se um espaço para exibição de uma breve descrição sobre o Deus, que pode ser exibida a medida que o jogador passa o mouse em cima de cada \textit{deus}.
\item No centro como dito anteriormente tem-se as três torres existentes sendo que as torres disponíveis possuem uma estrela para indicar que as mesmas são selecionáveis, caso a torre esteja bloqueada, um \textit{\textbf{"X"}} ocupa o lugar da estrela. Há também neste box, abaixo da figura das torres uma representação em \textit{gold} alcançada pelas torres deste tipo.
\item No box do conto inferior direito tem-se a torre selecionada, esta torre fica com um tom mais claro no menu central.
\item No canto superior esquerdo tem-se a barra de \textit{HP}, do jogador.
\item Finalizando tem-se no canto superior direito, as informações referentes ao \textit{gold} total alcançado.
\end{itemize}

\begin{figure}[!htp]
\centering
\includegraphics[scale=0.4]{res/gameplay.jpg}
\caption{Hud}
\label{HUD}
\end{figure}

\newpage

\section{Personagens Principais}

\subsection{Deuses}
Como dito nas secções anteriores, haverão 3 Deuses:

{\large \textbf{Zeus}}: Zeus é um Deus imortal considerado como o pai dos Desuse e dos homens, tido como o mais poderoso dentre os Desuses do \textit{Olimpo}. É quem exerce a ordem e domínio sobre os outros Deuses, é o Deus dos \textbf{Céus, Raios e Relâmpagos}, tem como conjunge a deusa \textit{Hera}, a deusa da maternidade. 

Diante da ameaça de invasão pelas \textit{waves} de monstros na terra, \textit{Zeus liderará} a tropa de Deuses responsáveis pela derrota destes monstros e vitoria da humanidade. 

Deste modo, sendo um Deus cujo as torres utilizam predominantemente raios para atacar, possuirá dano maior em inimigos do tipo fly.
\begin{figure}[!htp]
\centering
\includegraphics[scale=0.25]{res/characters/zeus.png}
\caption{Imagem conceitual de Zeus}
\label{zeus}
\end{figure}

Zeus é possui a aparência de um senhor sábio, extremamente poderoso e respeitável.

Para a caracterização do personagem foi adotada a técnica de \textit{Pixel-Arte}, uma vez que a equipe de artistas já possuem experiência neste tipo de técnica, otimizando  assim o tempo necessário para o desenvolvimento dos mesmos. Abaixo segue a imagem do personagem com a arte finalizada.

\begin{figure}[!htp]
\centering
\includegraphics[scale=2]{res/characters/zeus_panel.png}
\caption{Pixel-Arte de Zeus}
\label{zeus}
\end{figure}

\newpage

{\large \textbf{Poseidon}}: Poseidon é o Deus supremo do mar, também conhecido como Deus dos terremotos, é um dos irmãos de \textbf{Zeus}, podendo ser tão forte quanto o próprio irmão, mas devido ao fato do irmão \textbf{Zeus} te-lo salvo da barriga de seus pai o Titan \textbf{Chronos}, Poseidon e Zeus selarão um acordo, no qual \textbf{Zeus} seria o Deus supremo do \textbf{Olimpo} e \textbf{Poseidon} manteria o poder absoluto sobre os mares e oceanos. Comandado por Zeus Poseidon utilizara seus poderes para destruir os monstros invasores e restaurar o equilíbrio entre os mundos.

Como suas torres utilizam o poder das águas para atacar, possuirá assim dano maior em  inimigos do tipo \textit{speed}.

\begin{figure}[!htp]
\centering
\includegraphics[scale=0.25]{res/characters/poseidom.png}
\caption{Exemplo conceitual de poseidom}
\label{poseidom}
\end{figure}

Para a caracterização deste personagens utilizou-se uma arte predominantemente azul, para simbolizar as águas.

\begin{figure}[!htp]
\centering
\includegraphics[scale=2]{res/characters/poseidon_panel.png}
\caption{Pixel-Arte de Poseidon}
\label{zeus}
\end{figure}

\newpage

{\large \textbf{Hades}}: Hades é o deus dos mortos e do mundo inferior, é o deus mais odiado pelos humanos, não tendo também nenhuma simpatia pelos mesmos, porém ao ser recrutado por \textbf{Zeus}, acaba se juntando ao seu exercito de Deuses e combatendo os monstros invasores.

Sendo um Deus que utiliza maldições para atacar, possuirá dano maior em inimigos do tipo tanker.

\begin{figure}[!htp]
\centering
\includegraphics[scale=0.25]{res/characters/hades.png}
\caption{Exemplo conceitual de hades}
\label{satiro}
\end{figure}

Para a caracterização do personagem optou-se por um tom mais sombrio, uma vez que o mesmo é o senhor do mundo dos mortos, assim definiu-se um tom mais escuro e com fogo, representando as chamas do submundo.

\begin{figure}[!htp]
\centering
\includegraphics[scale=2]{res/characters/hades_panel.png}
\caption{Pixel-Arte de Poseidon}
\label{zeus}
\end{figure}

\newpage


\subsection{Torres}
Serão três torres para cada Deus, sendo divididas em dois níveis, para subir de nível e liberar a torre de tier superior é necessário conseguir uma quantia \textit{gold}. A progressão de níveis entre as torres esta descrito na secção economia, detalhando o valor de cada uma das torres bem como os valores para evoluir as mesmas.

\paragraph{{\Large Zeus}: As torres associadas à Zeus, terão como ataque raios. Sendo que os ataques, possuem um alto poder, porém demoram um pouco para serem lançados.}

\begin{figure}[!htp]
\centering
\includegraphics[scale=1.3]{res/characters/zeus_tower.png}
\caption{torre de Zeus}
\label{satiro}
\end{figure}

\paragraph{{\Large Hades}: As torres associadas à Hades, atacarão com bolas de fogo. Sendo que os ataques, possuem um poder mediando e uma frequência rápida.}

\paragraph{{\Large Poseidon}: As torres associadas à Poseidon, atacarão com ondas de água. Estas torres possuem o menor dano no ataque entre as torres, porém elas ao atacarem colocam os inimigos em velocidade \textit{slow} possuem um poder mediando e uma frequência rápida.}

\begin{figure}[!htp]
\centering
\includegraphics[scale=1.3]{res/characters/poseidon_tower.png}
\caption{torre do poseidon}
\label{satiro}
\end{figure}



\section{Inimigos}
\subsection{Waves} 

Haverão diversas \textit{waves}, onde os monstros estarão divididas entre 4 categorias, sendo elas normal, tanker, fly e  speed, a principio não haverá um monstro o \textit{Boss}, caso o jogo evolua poderia-se pensar em um \textit{boss}.

As \textit{waves} sempre nascerão no mesmo lugar, que pode ser alternado somente entre as fases do jogo, ao nascer as \textit{waves} farão sempre o mesmo caminho seguindo a trilha pre-definida e indicada por uma cor diferente no mapa, mantendo uma velocidade constante da sua categoria de \textit{waves}. 

As \textit{waves} não interagirão no cenário ou seja não atacarão as torres nem possuirão um mecanismo para evitar ataques das torres, ou seja ao entrar no campo de visão de ataque de uma torre as \textit{waves} serão atacadas, porém caso a torre já esteja atacando um monstro esta continuara atacando a este monstro enquanto o monstro tiver no campo de visão da torre, podendo atacar o novo monstro somente quando o monstro anterior sair do campo de visão da torre.  

Para a derrota da \textit{wave}, serão atribuídos \textit{golds} para cada monstro derrotado e caso toda a wave seja derrotada, sem que nenhum mostro consiga completar o percurso chegando ao outro lado do mapa, um bônus extra em \textit{gold} será atribuído ao jogador. 

Cada monstro terá uma quantia de \textit{HP} especifica variando de acordo a sua categoria, no qual eles somente serão derrotados ao zerar este \textit{HP}, e caso isto ocorrer eles automaticamente somem do mapa e uma quantia de \textit{gold} é atribuída ao jogador.

As \textit{waves} terão uma quantia de inimigos randômica, podendo variar entre as fases, e os monstros serão inseridos randomicamente nas \textit{waves}.

Devido ao tempo de desenvolvimento de jogo não haverá diferenciação de inimigos por fases no jogo, isto acarretaria um esforço gigantesco na criação de personagens uma vez que por fase são 4 categorias diferentes de inimigos, ocasionando um grande trabalho inicialmente desnecessário. 

\subsection{categorias}
Descrição dos inimigos pertencentes as \textit{waves}, como o jogo é um \textit{Tower Defense} estes inimigos serão inseridos randomicamente nas fases do jogo, no caso inicialmente serão desenvolvidas somente três níveis, que alternam somente o mapa e cenário do jogo, assim não haverá diferenciação dos inimigos nos níveis que compõem o jogo.

\textbf{{ {\large Normal}}} – na primeira categoria de inimigos tem-se os inimigos do tipo normal, estes não terão nenhum beneficio sendo bem equilibrados e aparecerão em maior quantidade, esta categoria de monstro será representada pela \textbf{Meduza}, que será como dito na mitologia grega uma mulher com cabeça de cobra e neste casso com um corpo de cobra abaixo da cintura.

\begin{itemize}
\item HP - 50
\item Def - 50
\item Sp - 50
\item Death - 50 golds.
\end{itemize}

\begin{figure}[!htp]
\centering
\includegraphics[scale=4]{res/characters/medusa.png}
\caption{Meduza do jogo}
\label{Meduza}
\end{figure}

\newpage

\textbf{{\large Tanker}} – Os \textit{tankers}, serão inimigos que possuirão um HP mais alto do que os outros monstros, sendo necessário vários ataques para causar danos, eles serão os \textit{cyclops}.
\begin{itemize}
\item HP - 150
\item Def - 50
\item Sp - 50
\item Death - 100 golds
\end{itemize}

\begin{figure}[!htp]
\centering
\includegraphics[scale=1]{res/characters/cyclop.png} 
\caption{cyclop sprite sheet}
\label{cyclops}
\end{figure}

\newpage

\textbf{{\large Fly}} – Os \textit{Fly}, serão inimigos que possuirão uma defesa elevada sendo necessário ataques fortes para causar dano, porém mantém um hp normal, estes inimigos são as \textit{harpias}.
\begin{itemize}
\item HP - 50
\item Def - 100
\item Sp -50
\item Deth - 75 golds.
\end{itemize}

\begin{figure}[!htp]
\centering
\includegraphics[scale=0.75]{res/characters/harpia.png} \quad
\includegraphics[scale=2]{res/characters/bat.png} 
\caption{harpia conceito}
\label{cyclops}
\end{figure}

\textbf{{\large Speed}} – Os \textit{Centauros}, serão os inimigos do tipo \textit{speed}, estes possuirão uma alta velocidade, porém um baixo hp.
\begin{itemize}
\item HP - 50
\item Def - 50
\item Sp -80
\end{itemize}

\begin{figure}[!htp]
\centering
\includegraphics[scale=0.4]{res/characters/centauro.jpg} \quad
\includegraphics[scale=3]{res/characters/centauro.png} 
\caption{Exemplo de centauros e pixel-art do centauro}
\label{cyclops}
\end{figure}

\newpage

\section{Saúde}

O jogador terá uma barra de vida com 50 HP, esta barra não conterá valores visíveis ao jogador, sendo uma longa barra vermelha no canto superior esquerdo, a qual terá o símbolo em formato de cruz vermelha, mundialmente conhecido como símbolo de saúde.

\begin{figure}[!htp]
\centering
\includegraphics[scale=0.3]{res/saude.png}
\caption{Demonstração da Saúde no mapa do jogo}
\label{Saúde}
\end{figure}
 
A barra de hp decairá a medida que os inimigos atravessem o mapa, sendo que cada tipo de inimigo tem uma pontuação diferente, tais valores são estes abaixo:

\begin{itemize}
 \item Norma = -2 
 \item Tanker = -5
 \item Speed = -3
 \item Fly = -3
 \end{itemize} 

Quando a saúde atingir um nível de 20\% a música será trocada para um ritmo mais tenso, para dar mais emoção ao jogo, também ocorrerá a troca da imagem de fundo para um tom mais avermelhado.

\begin{figure}[!htp]
\centering
\includegraphics[scale=0.3]{res/danger.png}
\caption{Demonstração da Saúde no mapa do jogo em um baixa hp}
\label{Saúde}
\end{figure}

\subsection{Restauração}
A saúde não será restaurada durante a partida, acontecendo somente quando o jogador passar de fase, quando ocorrer o jogador terá sua barra de \textit{HP} restaurada em 30 HP, ou seja se o jogador terminou a primeira fase com somente 10 de \textbf{HP}, ele começara a segunda fase com \textbf{40 HP}, e assim sucessivamente nas demais fases do jogo.  

O estado do jogo não será salvo, somente os mapas abertos serão salvos no jogo.

\subsection{Game Over}
O \textit{game Over} só ocorrera se a barra de \textbf{HP} for zerada, assim quando ocorrer uma grande mensagem cobrira a tela do jogo com a frase de \textbf{Game Over}, e as opções para uma nova partida ou para fechar o jogo.

\begin{figure}[!htp]
\centering
\includegraphics[scale=0.3]{res/game_over.png}
\caption{Tela de Game Over}
\label{Game Over}
\end{figure}
\newpage

\section{Economia}

A única remuneração do jogo será em \textit{gold}, o qual terá como intuito incrementar as torres, não havendo nenhum tempo especifico para a compra de torres, exceto o tempo inicial de jogo, que o jogador seleciona as torres desejadas antes do jogo iniciar de fato, assim as compras podem ser realizadas em qualquer momento da partida.

Os \textit{golds}, são acumulativos ao decorrer das fases, sendo perdidos somente no \textit{game over}.

Toda pontuação do jogador será obtido através dos monstros derrotados, quanto mais forte for o inimigo  maior será a recompensa em ouro, sendo que cada um dos monstros possuem uma pontuação diferente, tal pontuação possui os seguinte valores:

\begin{itemize}
 \item Norma = 2 Golds
 \item Tanker = 5 Golds
 \item Speed = 3 Golds
 \item Fly = 3 Golds
 \end{itemize} 
 
Como as torres tem dois \textit{Tiers}, haverá valores diferentes para as torres de cada \textit{Tiers}, assim para as torres de \textit{Tier 1}, ou seja de nível mais baixo, serão necessários 20 \textit{Golds} para aquisição das mesmas, já para as torres de \textit{Tier 2} será necessária a quantia de 40 \textit{golds} para aquisição das mesmas, caso estas estejam disponíveis, uma vez que para liberar o acesso a torres de \textit{Tier 2} é necessário alcançar 40 golds na torre de \textit{Tier 1} respectiva, por exemplo para liberar as torres de \textit{Tier 2} do \textit{Deus} \textbf{Zeus} é necessário alcançar a marca de 40 golds nas torres de \textit{Tier 1} do \textbf{Zeus}, e assim respectivamente nos outros \textbf{Deuses}.

Não haverá diferenciação nos valores entre as torres por Deuses, somente entre os \textit{Tiers} das torres, assim cada Deus terá torres nos dois \textit{Tiers} existentes.
 
No inicio do jogo, tem-se disponível a quantia de ouro suficiente para a compra de somente duas torres de \textit{Tier 1}, ou seja 40 golds, as quais podem ser de investidas nas torres de qualquer um dos Deuses.  

Caso o jogador se arrependa de ter posicionado uma torre em um local ineficiente, ou por qualquer outro motivo que seja, ele terá a opção de recupera metade do \textit{gold} investido, retirando a torre do local, ou sejá uma torre que custa 20 golds, terá 10 \textit{golds} recuperados.
 
 
O \textit{gold} será representado por um imagem de ouro e uma quantia ao lado da imagem, essa representação será posicionada no canto superior direito da tela. Também tem-se uma imagem contendo o valor de \textit{gold} alcançado por cada torre existente, este valor fica no HUD abaixo da torre em questão.

Resumindo tem-se a seguinte lista com os valores das torres:
\begin{itemize}
 \item Tier 1 = 20 Golds.
 \item Tier 2 = 40 Golds.
 \item Liberar Tier 2 = 40 Golds em uma torre de Tier 1.
 \item Recuperar torre = perca de metade dos golds.
\end{itemize} 

\begin{figure}[!htp]
\centering
\includegraphics[scale=1.25]{res/gold.png}
\caption{Representação Gold}
\label{Tela Equip}
\end{figure}

\begin{figure}[!htp]
\centering
\includegraphics[scale=1.0]{res/torresGold.png}
\caption{Representação Gold associado as torres.}
\label{Tela Equip}
\end{figure}

\newpage

\section{Pontuação}
Como descrito na secção Economia, o jogo terá somente um sistema de pontuação, em \textit{golds}, não havendo demais itens de suporte ao jogador.

Devido ao tempo de projeto a equipe decidiu concentrar esforços no desenvolvimento do \textit{gameplay} do jogo, portanto definiu-se que não seria criado um sistema de \textit{Ranking}, assim os golds pertencerão jogador durante a partida, sendo perdidos ao final do jogo, futuramente se for possível, após o final do desenvolvimento do \textit{gameplay} do jogo, no tempo de projeto da disciplina, pode-se criar um sistema local de \textit{Ranking}, porém como dito anteriormente este \textit{Ranking} não é prioridade, caso seja desenvolvido deve ser tratado como um extra ao projeto.  

\begin{figure}[!htp]
\centering
\includegraphics[scale=0.3]{res/pontuacao.png}
\caption{Demontração da pontuação no mapa do jogo}
\label{pontuacao}
\end{figure}
 

\section{Esquema de controle e interface com o usuário}

O jogador movimentará somente o mouse, utilizando o mesmo para colocar as torres na posição desejada, sendo que o mesmo seleciona a torre desejada no menu inferior, o mouse indica que a torre foi selecionada, com a torre sobrepondo o cursor do mouse.

Um circulo em volta do cursor será formado, para indicar o alcance de ataque da torre, caso o local para adicionar a torre sejá inválido este circulo ficará avermelhado.  

\newpage

\section{Front End}

O \textit{Front End} do jogo terá  quatro telas principais, sendo elas, menu de \textit{Abertura}, \textit{Fases}, \textit{Gameplay} e \textit{Créditos}, além de uma tela auxiliar de carregamento \textit{loading screen}. A seguir uma representação das telas do jogo.

\subsection{Tela de Abertura}
A tela de abertura do jogo dará acesso para outras duas telas, que são: \textit{Gameplay} e \textit{Credits}.

Nesta tela haverá a imagem do jogo ao fundo, e um menu com as seguintes opções para o jogador: 
\textit{Play Game} que levará o jogador para a tela de \textit{Gameplay}.
\textit{Crédits} para exibição dos créditos.
\textit{Exit} que fechara o jogo.

\begin{figure}[!htp]
\centering
\includegraphics[scale=0.4]{res/abertura.png}
\caption{Exemplo de Abertura}
\label{Abertura}
\end{figure}

\subsection{Tela de Fases}
Nesta tela o jogador escolhe as fases que desejá jogar, ao inicio do jogo somente a primeira fase encontra-se disponível, para abrir as subsequentes é necessário sobreviver nas fases anteriores.

As fases abertas estarão com uma imagem do mapa visível, já as imagens fechadas estarão cinzas.

Nesta tela, haverá também um ícone para voltar a tela anterior.  

Ao selecionar uma fase, aparecerá uma mensagem pedindo a confirmação da escolha e o jogador será redirecionado para a página de \textit{Gameplay}, caso concorde.

\begin{figure}[!htp]
\centering
\includegraphics[scale=1]{res/fases.jpg}
\caption{Exemplo da tela de fases}
\label{Fases}
\end{figure}

\newpage

\subsection{Tela de Gameplay}

Tela principal do jogo, nesta tela tem-se um menu na parte inferior da tela, com as informações do Deus no quadro inferior esquerdo, ao clicar no Deus sua torre é selecionada e aparece no quadro inferior direito.
No quadro inferior central tem-se as informações sobre as torres, sendo que as torres disponíveis ficam com a imagem da torre visível já as torres indisponíveis ficam com um "\textbf{x}", há também um pequeno quadro indicando quanto de \textit{gold} falta para evoluir a torre e liberar o próximo tier.

Para adicionar uma nova torre o jogador deverá clicar na torre no box inferior direito e arrasta-lá no mapa até a posição desejado.

O mapa fica no centro da tela, ocupando o máximo de área possível.

Na parte superior direita haverá um relógio contendo o tempo restante de fase.

Na parte superior central tem-se informações sobre as \textit{waves}.

\begin{figure}[!htp]
\centering
\includegraphics[scale=0.5]{res/gameplay.jpg}
\caption{Exemplo de Gameplay}
\end{figure}

\newpage
\subsection{Tela de Créditos}

Na tela de \textit{Cr} exibe-se o nome e a função de cada um dos integrantes da equipe, sendo de \textit{Desnvolvimento}, \textit{artes} e \textit{música}.

\newpage

\subsection{Logo}

A logo a seguir o logo do SomTD].
\begin{figure}[!htp]
\centering
\includegraphics[scale=1.1]{res/logo.png}
\caption{Logo do SomTD}
\label{Logo do SomTD}
\end{figure}

A logo a seguinte representa a API utilizada durante o desenvolvimento.

\begin{figure}[!htp]
\centering
\includegraphics[scale=0.3]{res/Sdl-logo.png}
\caption{Logo da SDL}
\label{Logo da SDL}
\end{figure}

\begin{figure}[!htp]
\centering
\includegraphics[scale=0.3]{res/classification.png}
\caption{Classificação Indicativa}
\label{Classificação Indicativa}
\end{figure}

A classificação indicativa definida foi de 12 anos, devido a possíveis traços de violência no jogo, . 


\newpage

\section{Músicas e Efeitos Sonoros}
\subsection{Músicas}
Para a composição da musica do jogo tem-se três categorias uma abertura, fases créditos. 

\begin{enumerate}
\item 
\textbf{Nome:} Musica de Abertura

\textbf{Descrição:} Esta musica será sinfônica e curta ocorrendo em \textit{loop}, inspirada em jogos famosos como \textit{Mario} e principalmente \textit{Zelda}, assim deve ser uma música "exuberante" e chamativa de modo a ser identificada.

\textbf{Utilização:} será utilizada na tela de abertura do jogo.

\item
\textbf{Nome:} Gameplay

\textbf{Descrição:} Esta musica tem como intuito deixar o jogador confortável durante o jogo, assim deve ter uma melodia tranquila uma vez que a mesma será reproduzida em \textit{loop} durante a partida, assim será uma musica semelhante a "musica de elevador", calma com poucos picos, para evitar que a mesma acabe por "competir" com os efeitos sonoros do jogo, mesmo que sejam poucos.  

\textbf{Utilização:} essa musica será utilizada durante todo o \textit{gameplay} do jogo.

\item
\textbf{Nome:} Créditos

\textbf{Descrição:} Musica a ser exibida durante os créditos, esta é uma música de comemoração portanto deve ser uma música alegre e de preferencia eletrônica. 

\textbf{Utilização:} Utilizada na tela de créditos.
\end{enumerate}

\subsection{Efeitos Sonoros}
{\LARGE \textbf{Deuses}}

\begin{enumerate}
\item 
\textbf{Nome:} Dublagem. 

\textbf{Descrição}: haverá uma dublagem relativa ao deus e quando um deus for selecionado uma a frase especifica será lançada.

\textbf{Utilização:} Será utilizada quando o jogador clicar na imagem de um deus no painel. Tem-se as seguintes frases relativas aos deuses:
\begin{itemize}
\item \textbf{Zeus}: "Eu sou o pai dos homens e dos Deuses".
\item \textbf{Hades}: "Temam o poder do deus do sub mundo".
\item \textbf{Poseidon}: "Eu sou o Deus supremo dos mares e dos Oceanos, temam o meu poder".
\end{itemize}

\item
\textbf{Nome:} Efeito ao selecionar o Deus.

\textbf{Descrição}: Ao clicar em um deus além da dublagem tem-se um efeito característico ao Deus associado, este efeito é rápido e com um volume baixo, uma vez que o jogador ira clicar varias vezes nos Deuses ao decorrer da partida, este efeito pode ser desligado, porém ao desliga-lo, desliga-se também a dublagem.

\textbf{Utilização:} Será utilizado ao clicar em um deus.
\begin{itemize}
\item \textbf{Zeus}: barulho de trovão.
\item \textbf{Hades}: ruídos de magia.
\item \textbf{Poseidon}: barulho de ondas.
\end{itemize}

\begin{LARGE}
\textbf{Waves}

\end{LARGE}
\item
\textbf{Nome:} Movimentação das waves

\textbf{Descrição:} tem-se um leve barulho de passos para os monstros (\textit{medusa, minotauro, cyclop} e um som de bater de assas para as harpias, estes sons devem ser baixos e acontecem principalmente quando uma nova \textit{wave} está entrando no cenário, diminuindo gradativamente na partida, para evitar uma poluição sonora no jogo.  

\textbf{Utilização:} utiliza-se este efeito quando as \textit{waves} entrarem no cenário.

\item
\textbf{Nome:} Morte dos monstros

\textbf{Descrição:} leve efeito semelhante a um "\textit{pou}", ou um barulho de fumaça. Este efeito é bem curto com aproximadamente 1 segundo. Todos os monstros compartilharão do mesmo efeito.

\textbf{Utilização:} na morte de um monstro.


{\LARGE \textbf{Torres}}

\item
\textbf{Nome:} Som de ataque.

\textbf{Descrição:} como cada Deus terá duas torres, tem-se duas sons de ataque diferentes um para cada torre, porém nestes três tipos de ataque acontecem variações de sons referentes ao deus associado a torre.

\textbf{Utilização:} 
\begin{itemize}
\item \textbf{Zeus}: Ao atacarem as torres terão variações de ataques com ruídos de raios e trovões.
\item \textbf{Hades}: Para as torres de \textit{Hades}, tem-se variações ruídos de magia, semelhantes ao barulho de fumaça dos filmes.
\item \textbf{Poseidon}: Para as torres do Poseidon, tem-se sons semelhantes ao barulho de água batendo no chão além de ondas de água.
\end{itemize}

{\LARGE \textbf{Cenário}}

\item
\textbf{Nome:} Tilintar de moedas

\textbf{Descrição:} um barulho de tilintar de moedas para indicar que houve alteração no \textit{gold} durante a compra de uma torre. 

\textbf{Utilização:} é utilizado quando o \textit{gold} do jogador é reduzido.

\item
\textbf{Nome:} Vitoria

\textbf{Descrição:} Uma dublagem semelhante a frase "\textit{You wim}" de jogos antigos, com um barulho de fundo semelhante a um "\textit{tam tanram}".

\textbf{Utilização:} é utilizada quando o jogador vence as \textit{waves}.

\item
\textbf{Nome:} Derrota

\textbf{Descrição:} Semelhante ao efeito de vitoria porém esta é inspirada na derrota de jogos como "God of War" ou "Mortal Kombat", sendo algo parecido com um tintilar de metais, além de uma frase parecida com "\textit{You Lose}".

\textbf{Utilização:} utilizada quando o jogador morre.

\item
\textbf{Nome:} Inimigo atravessando o campo.

\textbf{Descrição:} Um chiado para representar que um inimigo chegou ao final do campo.
\textbf{Utilização:} utilizado quando um inimigo atravessa e chega no final do campo.

\end{enumerate}
\section{Requisitos tecnológicos}

\begin{figure}[!htp]
\begin{center}
  \includegraphics[scale=0.04]{res/audacity.png} \quad
  \includegraphics[scale=0.3]{res/adobe_illustrator.png} \quad
  \includegraphics[scale=0.25]{res/Sdl-logo.png} \quad
  \includegraphics[scale=0.3]{res/Sublime_Text_Logo.png} \quad
  \includegraphics[scale=0.2]{res/cpp.png} \quad
  \includegraphics[scale=0.13]{res/git.png} \quad
  \includegraphics[scale=0.5]{res/linux.png} \quad
\caption{Recursos Tecnológicos} \label{gdimotes}
\end{center}
\end{figure}

\section*{Informações de contato}
{\Huge \textbf{strifeofmythology@gmail.com}}

\end{document}
