%%%%%%%%%%%%%%%%%%%%%%%%%%%%%%%%%%%%%%%%%
% Thin Sectioned Essay
% LaTeX Template
% Version 1.0 (3/8/13)
%
% This template has been downloaded from:
% http://www.LaTeXTemplates.com
%
% Original Author:
% Nicolas Diaz (nsdiaz@uc.cl) with extensive modifications by:
% Vel (vel@latextemplates.com)
%
% License:
% CC BY-NC-SA 3.0 (http://creativecommons.org/licenses/by-nc-sa/3.0/)
%
%%%%%%%%%%%%%%%%%%%%%%%%%%%%%%%%%%%%%%%%%

%----------------------------------------------------------------------------------------
%   PACKAGES AND OTHER DOCUMENT CONFIGURATIONS
%----------------------------------------------------------------------------------------

\documentclass[11pt]{article} % Font size (can be 10pt, 11pt or 12pt) and paper size (remove a4paper for US letter paper)

\usepackage[utf8]{inputenc} % Set utf8 code
\usepackage[protrusion=true,expansion=true]{microtype} % Better typography
\usepackage[portuguese]{babel}
\usepackage{graphicx} % Required for including pictures
\usepackage{wrapfig} % Allows in-line images
\usepackage[pagebackref]{hyperref}

\usepackage{mathpazo} % Use the Palatino font
\usepackage[T1]{fontenc} % Required for accented characters

\usepackage{wallpaper}
\usepackage[font={color=white,bf},figurename=Fig.,labelfont={it}]{caption}
\usepackage{lipsum, xcolor, etoolbox, footmisc, bigfoot}

\usepackage{tabu}
\hypersetup{
    colorlinks=false,
    pdfborder={0 0 0},
}

\setcounter{secnumdepth}{5}
\setcounter{tocdepth}{5}

\linespread{1.05} % Change line spacing here, Palatino benefits from a slight increase by default

\makeatletter
\renewcommand\@biblabel[1]{\textbf{#1.}} % Change the square brackets for each bibliography item from '[1]' to '1.'
\renewcommand{\@listI}{\itemsep=0pt} % Reduce the space between items in the itemize and enumerate environments and the bibliography

\renewcommand{\maketitle}{ % Customize the title - do not edit title and author name here, see the TITLE block below


\begin{center} % Right align
{\LARGE\@title} % Increase the font size of the title

\vspace{20pt} % Some vertical space between the title and author name

\end{center}
}

\patchcmd{\ps@plain}{\thepage}{\textcolor{white}{\thepage}}{}{}
\makeatother

\begin{document}

\ThisTileWallPaper{\paperwidth}{\paperheight}{res/wallpaper_header.jpg}
\color{white}
\pagestyle{plain}
\def\footnotelayout{\color{white}}
\renewcommand\thefootnote{\textcolor{white}{\arabic{footnote}}}
\begin{titlepage}
 \vfill
  \begin{center}
   {\textbf{{{\Huge  Strife Of Mythology Tower Defense}}}}\\[6cm]


   {{\huge Game Design Document}}\\[6cm]

   \hspace{.45\textwidth} %posiciona a minipage
  \vfill

\vspace{2cm}

\large \textbf{Brasília}

\large \textbf{Abril de 2016}
\end{center}
\end{titlepage}
\newpage
\color{black}
\tableofcontents

\newpage

%----------------------------------------------------------------------------------------
%   DOC BODY
%----------------------------------------------------------------------------------------

\TileWallPaper{\paperwidth}{\paperheight}{res/wallpaper_body.jpg}
\color{white}

\section*{Tabela de Revisão}


\begin{table}[h]

  \taburulecolor{white}
  \color{white}
\begin{tabu}{|l|l|p{60mm}|l|}

\hline 
\textbf{Versão}     & \textbf{Data}     & \textbf{Descrição}                              			& \textbf{Autor}    \\ \hline
0.1                 & 05/03/16        & Objetivo / História / Controles                    			& Marcelo Martins     \\ \hline
0.2                 & 06/03/16        & Requisitos Tecnológicos                            			& Marcelo Martins     \\ \hline
0.3                 & 13/03/16        & Aplicando correções propostas                      			& Marcelo Martins     \\ \hline
0.4                 & 12/04/16      & Adição dos Personagens / Refinamento dos itens anteriores   	& Marcelo Martins     \\ \hline
0.5                 & 29/04/2015        & Rascunho das telas				                   			& Jônntas Lennon     \\ \hline
0.5                 & 10/05/2015        & telas / Personagens				                   			& Jônntas Lennon     \\ \hline

\end{tabu}
\end{table}

\newpage

\section{Objetivo do jogo}

O \textit{\textbf{SoMTD}} é um jogo de estilo\textit{ Tower Defense}, onde o objetivo é impedir que as \textit{waves} de monstros avancem pela caminho utilizando torres que atacam estas \textit{waves}, estas torres são pertencentes a três Deuses.
 A medida em que as \textit{waves} são vencidas, elas ficam mais fortes e difíceis de derrotar. Nisto haverá um recurso que servirá para evoluir e construir novas torres, o qual será obtido conforme os monstros forem derrotados.

\section{Glossário}
\begin{itemize}
\item \textbf{Wave}: Inimigos que nascem em conjunto em momentos do jogo.
\item \textbf{Caminho}: Caminho pré-determinado que será seguido pelas \textit{waves}. Caso um inimigo chegue ao fim da trilha, o jogador sofre uma punição.
\item \textbf{Tower Defense}: Um estilo de jogo onde um jogador deve impedir unidades inimigas de completarem seu objetivo utilizando torres.
\item \textbf{Health Point(HP)}: Atributo referente a vida \textit{player}.
\end{itemize}
 
\section{Conceito do jogo}

As torres pertencem  três Deuses sendo eles:
\begin{itemize}
\item Hades.
\item Zeus.
\item Poseidom.
\end{itemize}

Cada Deus terá um conjunto de Torres a seu dispor, divididas em 2 categorias sendo que cada categoria possuirá uma singularidade em relação as demais, possuindo assim vantagens em relação a alguns tipos de monstros e desvantagens em relação a outros tipos de monstros.

Ao iniciar o jogo, o jogador terá uma quantidade de ouro suficiente para adquirir e colocar torres de forma estratégica, tentando evitar a passagem dos inimigos pela trilha, sendo que no início do jogo apenas um tipo de torre por Deus estará disponível, tais torres podem ser evoluídas ao passo que a torre selecionada consiga destruir os monstros. 

A mudança de cenário ocorrerá ao decorrer do jogo, a medida em que as \textit{waves} forem derrotadas o jogador alcançará pontos de performance, que ao completar a barra referente a estes pontos, ocorrerá uma troca de era, havendo assim a mudança de fase e consequentemente de cenário. 

O final do jogo pode-se dar de duas formas:

\begin{enumerate}
\item O jogador posicionou as torres da forma que ele acreditava ser a forma mais estratégica, contudo não era a melhor forma, e ele não conseguiu suportar a \textit{wave},fazendo com que os inimigos cheguem ao final da trilha, com isso ele irá acumular penalidades, atingindo o máximo de penalidades aparecera a mensagem de \textit{Game Over};
\item O jogador consegue posicionar as torres da forma mais estratégica, conseguindo assim, suportar todas as \textit{waves} impostas, caso isso ocorra, ele irá receber uma mensagem de \textit{You Win}.

\end{enumerate}

\section{Gameplay}

\subsection{Progressão do Jogo}

\subsection{Timer}

\section{Mecânicas do Jogo}

\subsection{Exploração Vertical}

\subsubsection{Movimentação}
A movimentação das waves será em um caminho pré-determinado, cabendo ao jogador apenas posicionar as torres com o mouse próxima a esse caminho, não havendo assim movimentação por parte do jogador principal.

\subsubsection{Resolução}
\newpage

\section{Telas}

\subsection{Recursos}

O jogo contará com 3 recursos distintos para realizar a progressão da base e dos personagens, sendo eles Data, Matéria e Energia.

\subsection{Estrutura}

\section{Personagens}

\section{Saúde}

\section{Telas}

\subsection{Tela de Abertura}

A tela de abertura do jogo dará acesso para outras três telas, que são: \textit{Play Game}, \textit{Options} e \textit{Credits}.

\begin{figure}[!htp]
\centering
\includegraphics[scale=0.25]{res/map_example.png}
\caption{Abertura}
\label{Abertura}
\end{figure}

\newpage

\subsection{Tela de Play Game}

A tela de \textit{Play Game} é a tela onde o jogador poderá criar um novo jogo ou carregar algum \textit{slot} salvo anteriormente.

\begin{figure}[!htp]
\centering
\includegraphics[scale=0.25]{res/map_example.png}
\caption{Play Game}
\label{Play Game}
\end{figure}

\subsection{Tela de Options}

A tela de \textit{Options} é a tela onde o jogador poderá configurar o volume do e a resolução do jogo da maneira que este achar mais adequado.

\begin{figure}[!htp]
\centering
\includegraphics[scale=0.25]{res/map_example.png}
\caption{Options}
\label{Options}
\end{figure}

\subsection{Tela de Credits}

Na tela de \textit{Credits} poderá ser observado o nome e a função dos integrantes do time de desenvolvimento da Tiamat.

\begin{figure}[!htp]
\centering
\includegraphics[scale=0.25]{res/map_example.png}
\caption{Credits}
\label{Credits}
\end{figure}

\newpage

\section{Câmera e HUD}

\subsection{Câmera}

A câmera do jogo será 2D estática em um mapa isométrico, como exemplificado pela imagem abaixo.
\begin{figure}[!htp]
\centering
\includegraphics[scale=0.25]{res/map_example.png}
\caption{Tela Barracks}
\label{Tela Barracks}
\end{figure}

\subsection{HUD}

O HUD (Heads-Up Display) será composto de um menu inferior o qual será dividido em três secções,, sendo elas:
\begin{itemize}
\item Um circulo contendo o Deus selecionado, ao clicar neste Deus, automaticamente ocorrerá a troca para o próximo Deus.
\item A direita do circulo tem-se uma barra de HP, que decairá quando as waves atravessarem o caminho, além das informações de gold disponíveis para incrementar as torres.
\item Finalizando tem-se as informações referentes aos Tiers das torres, associadas aos Deuses.
\end{itemize}

\begin{figure}[!htp]
\centering
\includegraphics[scale=0.25]{res/map_example.png}
\caption{Tela Barracks}
\label{Tela Barracks}
\end{figure}

Na tela \textit{é apresentado} todos os equipamentos no qual o personagem poderá utilizar durante o combate que se passarão durante a exploração vertical.

\begin{figure}[!htp]
\centering
\includegraphics[scale=0.25]{res/map_example.png}
\caption{Tela Equip}
\label{Tela Equip}
\end{figure}

\newpage

\section{Personagens Principais}

\subsection{Deuses}
\subsection{Torres}
\subsection{Waves}


\section{Colecionáveis}

\section{Esquema de controle e interface com o usuário}

O jogador movimentará somente o mouse, utilizando o mesmo para colocar as torres na posição desejada, sendo que o mesmo seleciona a torre desejada no menu inferior, o mouse indica que a torre foi selecionada, com a torre sobrepondo o cursor do mouse.
Para indicar o alcance de ataque da torre, um circulo em volta do cursor será formado, caso o local para adicionar a torre sejá inválido este circulo ficará avermelhado.  


\begin{figure}
\begin{center}
  \includegraphics[scale=0.2]{res/torreSelecionada}\caption{Exemplo de direcionamento de Torres} \quad
  \includegraphics[scale=0.2]{res/localInvalido.png}\caption{Exemplo posicionamento invalido}  \quad
  \includegraphics[scale=0.3]{res/mouse.jpg}\caption{Mouse} \quad
\end{center}
\end{figure}

\newpage

\section{Front End}

O \textit{Front End} o jogo terá terá cinco telas principais, sendo elas, menu de \textit{Abertura}, \textit{Fases}, \textit{Gameplay} e \textit{Créditos}. A seguir uma representação das telas do jogo.

\subsection{Menu de Abertura}
Nesta tela haverá quatro opções para o jogador,\textit{New Game} que levará o jogador para a tela de \textit{Gameplay} e \textit{Exit} que fechara o jogo, e \textit{Crédits} para exibição dos créditos.

\begin{figure}[!htp]
\centering
\includegraphics[scale=0.4]{res/abertura.png}
\caption{Exemplo de Abertura}
\label{Logo da Tiamat}
\end{figure}

\subsection{Fases}
Nesta tela o jogador escolhe as fases que desejá jogar, ao inicio do jogo somente a primeira fase encontra-se disponível, para abrir as subsequentes é necessário sobreviver nas fases anteriores.

As fases abertas estarão com uma imagem do mapa visível, já as imagens fechadas estarão cinzas.

Nesta tela, haverá também um ícone para voltar a tela anterior.  

Ao selecionar uma fase, aparecerá uma mensagem pedindo a confirmação da escolha e o jogador será redirecionado para a página de \textit{Gameplay}, caso concorde.

\begin{figure}[!htp]
\centering
\includegraphics[scale=1]{res/fases.jpg}
\caption{Exemplo da tela de fases}
\label{Logo da Tiamat}
\end{figure}

\subsection{Gameplay}
Tela principal do jogo, nesta tela tem-se um menu na parte inferior da tela, com as informações do Deus, e das torres, bem como a vida do jogador.

Na parte superior direita haverá um relógio contendo o tempo restante de fase.

Na parte superior central tem-se informações sobre as \textit{waves}.

\begin{figure}[!htp]
\centering
\includegraphics[scale=0.8]{res/gameplay.png}
\caption{Exemplo de Gameplay}
\label{Logo da Tiamat}
\end{figure}

\newpage

A logo a seguir o logo dos responsáveis pelo desenvolvimento do jogo.

\begin{figure}[!htp]
\centering
\includegraphics[scale=1.1]{res/logo.png}
\caption{Logo da Tiamat}
\label{Logo da Tiamat}
\end{figure}

A logo a seguinte representa a API utilizada durante o desenvolvimento.

\begin{figure}[!htp]
\centering
\includegraphics[scale=0.3]{res/Sdl-logo.png}
\caption{Logo da SDL}
\label{Logo da SDL}
\end{figure}

\begin{figure}[!htp]
\centering
\includegraphics[scale=0.3]{res/classification.png}
\caption{Classificação Indicativa}
\label{Classificação Indicativa}
\end{figure}

A classificação indicativa definida foi de 12 anos, devido a possíveis traços de violência no jogo, . 


\newpage

\section{Requisitos tecnológicos}

\begin{figure}[!htp]
\begin{center}
  \includegraphics[scale=0.04]{res/audacity.png} \quad
  \includegraphics[scale=0.3]{res/adobe_illustrator.png} \quad
  \includegraphics[scale=0.25]{res/Sdl-logo.png} \quad
  \includegraphics[scale=0.3]{res/Sublime_Text_Logo.png} \quad
  \includegraphics[scale=0.2]{res/cpp.png} \quad
  \includegraphics[scale=0.13]{res/git.png} \quad
  \includegraphics[scale=0.5]{res/linux.png} \quad
\caption{Recursos Tecnológicos} \label{gdimotes}
\end{center}
\end{figure}

\section*{Informações de contato}
{\Huge \textbf{strifeofmythology@gmail.com}}

\end{document}
